\documentclass[a4paper,12pt]{article} % This defines the style of your paper

\usepackage[top = 2.5cm, bottom = 2.5cm, left = 2.5cm, right = 2.5cm]{geometry} 
\usepackage[utf8]{inputenc} %utf8 % lettere accentate da tastiera
\usepackage[english]{babel} % lingua del documento
\usepackage[T1]{fontenc} % codifica dei font

\usepackage{multirow} % Multirow is for tables with multiple rows within one 
%cell.
\usepackage{booktabs} % For even nicer tables.

\usepackage{graphicx} 

\usepackage{setspace}
\setlength{\parindent}{0in}

\usepackage{float}

\usepackage{fancyhdr}

\usepackage{caption}
\usepackage{amssymb}
\usepackage{amsmath}
\usepackage{mathtools}
\usepackage{color}

\usepackage[hidelinks]{hyperref}
\usepackage{csquotes}
\usepackage{subfigure}

\pagestyle{fancy}

\setlength\parindent{24pt}

\fancyhf{}

\lhead{\footnotesize Deep Learning Lab: Assignment 2}

\rhead{\footnotesize Giorgia Adorni}

\cfoot{\footnotesize \thepage} 

\begin{document}
	

	\thispagestyle{empty}  
	\noindent{
	\begin{tabular}{p{15cm}} 
		{\large \bf Deep Learning Lab} \\
		Università della Svizzera Italiana \\ Faculty of Informatics \\ \today  \\
		\hline
		\\
	\end{tabular} 
	
	\vspace*{0.3cm} 
	
	\begin{center}
		{\Large \bf Assignment 2: Convolutional Neural Network}
		\vspace{2mm}
		
		{\bf Giorgia Adorni (giorgia.adorni@usi.ch)}
		
	\end{center}  
}
	\vspace{0.4cm}

	%%%%%%%%%%%%%%%%%%%%%%%%%%%%%%%%%%%%%%%%%%%%%%%%
	%%%%%%%%%%%%%%%%%%%%%%%%%%%%%%%%%%%%%%%%%%%%%%%%
	
	\section{Introduction}
	The scope of this project is to implement a convolutional neural network to 
	classify the images in the CIFAR-10 dataset.
	
	First of all, the original training set has been shuffled and divided into 
	train and validation sets, with $49000$ and $1000$ images respectively.
	
	A certain preprocessing has been applied to the data, in particular, the 
	pixel values, initially comprised between 0 and 255, have been rescaled 
	between 0 and 1 and some binary assignment matrices have been created to 
	represent the class assignments, which were integers between 0 and 9. 
	
	The architecture of the convolutional neural network follow the 
	instructions provided, as well as the hyper-parameter values.
	In the training phase, different batches were used, containing different 
	examples for each epoch.
	
	\section{Once per epoch, document the classification accuracy on the 
	validation set. The final accuracy should be close to 70\%.}
	
	\section{document the evolution of the validation accuracy with dropout}
	
	\section{hyperparameter settings}

	\section{compute the test set accuracy and document the result.} 

\end{document}
